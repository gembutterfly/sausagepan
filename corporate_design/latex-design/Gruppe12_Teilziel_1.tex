\documentclass[10pt,a4paper,notitlepage]{report}
\usepackage[utf8]{inputenc}
\usepackage{amsmath}
\usepackage{amsfonts}
\usepackage{amssymb}
\usepackage{graphicx}
\usepackage{xcolor}
\usepackage{geometry}
\geometry{a4paper, top=15mm, left=25mm, right=25mm, bottom=25mm, headsep=10mm, footskip=10mm}
\pagestyle{empty} %keine Kopf-/Fußzeile
\author{Sausage Pan}
\begin{document}
	%Farbdefinierung
	\definecolor{orange}{HTML}{F67800}
	\definecolor{hellorange}{HTML}{FFAD41}
	\definecolor{schwarz}{rgb}{0,0,0}
	%Stildefinitionen!!Wichtig!!
	\newcommand{\Eins}[1]{\color{orange}\textbf{{\Large#1}}} %Überschrift 1. Ordnung
	\newcommand{\Zwei}[1]{\color{orange}\textbf{{\large#1}}} %Überschrift 2. Ordnung
	\newcommand{\Drei}[1]{\color{orange}{\normalsize#1}} %Überschrift 3. Ordnung
	\newcommand{\Text}{\color{schwarz}} %normaler Fließtext
	\newcommand{\Fusszeile}
	{\textit{{\footnotesize Eckert, Georg - Roscher, Philipp - Krien, Alexandra - Sinakov, Sergej - Blasberg, Bettina - Groß, Stephanie Sara}}} %Fußzeile immer am Ende der Seite einfügen!
	%Randstreifen
	\marginpar{\vspace{3.0mm} \color{orange}\rule{0.8mm}{53.3mm} \\[3mm] \color{hellorange}\rule{0.8mm}{170mm}}
	%Header-Bild
	\begin{center}
		\includegraphics[width=160mm]{header2}
	\end{center}
	%Eigentlicher Inhalt :D
	\color{orange} \textit{Gruppenname:} \color{schwarz}sausage pan \color{orange} \textit{Gruppenleiter:} \color{schwarz}Georg Eckert   \color{orange} \textit{Protokollant:} \color{schwarz}Philipp Roscher\\
	\\
	\\
	\Eins{Spielthemen:}\\
	\\
	\\
	\Zwei{Sonnensystem}\\
	\\
	\Text
		Ein Außerirdischer Weltenreisender wird aus dem Kryoschlaf geweckt und vom
	Bordcomputer seines Schiffs auf ein nahe liegendes Sonnensystem mit einem
	bewohnten Planeten hingewiesen. Die Informationssysteme des Planeten wurden
	bereits angezapft und erste Informationen werden dem Astronauten präsentiert.
	Er durchquert die Oortsche Wolke und passiert nach und nach die Planeten.
	Nach Zerstörung eines Planeten und der zugehörigen Monde, werden ihm
	verschiedene Informationen zu diesen angezeigt. Die verbleibende Entfernung zur
	Sonne wird ständig angezeigt. Der Spieler lernt so den groben Aufbau des
	Sonnensystems und erlangt durch Zerstörung von Monden und Kometen extra
	Informationen. Ziel ist es die Datenbank des Schiffes zu 100\% zu füllen.
	Das Spiel soll als klassischer Weltraum-Sidescroller oder Top-Down-Scroller
	mit Comic-Rahmen, wo Boardcomputer und Außerirdischer dargestellt werden,
	umgesetzt werden.\\
	\\
	\underline{Zielgruppe:} 5. und 6. Klasse an Realschulen und Gymnasien\\
	\\
	\\
	\Zwei{Farbabenteuer}\\
	\\
	\Text
		In diesem Lernspiel soll es um grundlegendes Verständnis zur Farbenlehre gehen. \\
	In einer Welt ohne Farbe herrschen Monotonie und drückende Stille, als eines Tages 
	fremdartige Fragmente vom Himmel fallen. Eines dieser landet vor den Füßen unseres 
	Protagonisten und strahlt in leuchtenden Farben. Auf der Suche nach den restlichen 
	Puzzlestücken macht der Spieler sich auf eine Reise durch die bunte Farbenwelt an 
	deren Ende die Zusammensetzung eines strahlenden Regenbogens steht.\\
		Der Spieler erkundet eine an ein Jump n Run angelehnte Welt, in welcher er nur 
	durch den Einsatz von Farben verschiedene Hindernisse und Gegner beseitigen kann. 
	Dabei sollen Farbkonzepte angewandt werden, denkbar wäre hierbei beispielsweise 
	der Einsatz von Komplementärfarben um sich Angreifern entgegen zu stellen oder 
	auch die Färbung von Türen in der Umgebungsfarbe ($\rightarrow$ Farbmischung) um sie passieren 
	zu können. Werkzeug stellen die Grundfarben dar, die der Spieler beliebig nutzen kann, 
	um die Spielwelt zu erkunden.\\
		Das Spiel soll optimaler Weise ohne Text gestaltet werden, um eine möglichst junge 
	Zielgruppe ansprechen zu können.\\
	\\
	\underline{Zielgruppe:} Schulanfänger\\
	\\
	\\
	\\
	\\
	\\
	\Fusszeile
\end{document}