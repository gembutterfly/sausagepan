\documentclass[10pt]{book}
\usepackage[ngerman]{babel}
\usepackage[latin1]{inputec}
\begin{document}
	\chapter{Setting}
	\section{Story}
	Im Labyrinth gibt es einen Schatz der durch einen Drachen beschützt wird. Das weiß die ganze Welt und versucht diesen unter ihrer Gewalt zu bekommen. Die mutigsten reisen von allen Teilen der Welt, um in das Labyrinth zu treten und Ruhm und Ehre zu erlangen. Dafür müssen sie gegen andere Spieler antreten und am Ende den Drachen besiegen.
	\section{Setting}
	Es können bis zu fünf Spieler an dem Spiel teilnehmen. Dafür werden zwei Gruppen und ein Game Master durch Zufall entschieden.
	Der Game Master ist der Drache und er muss den Schatz vor den anderen zwei Gruppen beschützen. Dafür hat er verschiedene Fähigkeiten: Er kann die ganze Karte sehen und an den Punkt wo er sich befindet, Schalter bestätigen, um Wände zu schieben oder durch geheime Wege gehen, die nur er sehen kann.
	Die zwei Teams bestehen aus jeweils zwei Spielern, einen Kämpfer und einen Heiler. Nachdem diese Aufteilung entschieden wurde, können die Teilnehmer ihre Spielfigur auswählen. Dabei konzentriert dich der Kämpfer eher um den Angriff gegen die anderen Gegner und der Heiler um die Gesundheit seines Teams.
	Nach der Auswahl fängt das Spiel im Labyrinth an. 
	Dieser wird zufällig generiert und ist somit bei jedem neuen Spielstart anders. Dort befinden sich die Gruppen an verschiedenen Stellen und laufen durch alle Wege, um den Schatz zu bekommen oder gegeneinander zu kämpfen.
	\chapter{Charaktere}
	Bei den Spielfiguren gibt es eine große Auswahl, denn sie sind nicht nur unter Heiler und Kämpfer unterteilt. Sie spezialisieren sich auch durch ihre Kampfposition: Nah-, Mittel- oder Fernkampf.
	\section{Der Drache} 
	Der Beschützer des Schatzes und der Herr des Labyrinths. Als Einzelgänger ist er um einiges stärker als ein Team. Seine Verteidigung ist auch um einig höher als bei den normalen Spieler.
	\section{Fernkampf}
	\subsection{Schamane}
	Ein Magier mit der Fähigkeit, sich von seinem Körper zu trennen und als Geist durch Wände zu gehen. Er kann sich nur bis zu einer Gewissen Distanz von seinem Körper fernhalten. Dieser kann im Spiel dennoch angegriffen werden.
	\subsubsection{Bogenschütze}
	Seine besondere Attacke sind die brennende Pfeile. Wenn er diese schießt, entstehen Flammen auf den Weg und blockiert somit seinen Gegner für gewisse Zeit den Pfad.
	\section{Mittelkampf}
	\subsection{Ninja}
	Neben den Angriffen mit seinen Shuriken und kurzen Schwertern, kann er sich auch für kurze Zeit unsichtbar machen. Aber das kostet ihm auch mehr Angriffspunkte.
	\subsection{Hexe}
	Auch eine Magierin, die mit ihren besonderer Tränke kämpft. Gegen ihre Gegner wirft sie Gifttränke, um sie zu schwächen und für sich oder ihren Partner benutzt sie Heiltränken. Ist die Situation besonders heikel, kann sie auf ihren Besen reiten und schnell von einem Ort fliehen.
	\section{Nahkampf}
	\subsection{Schwertkämpfer}
	Er kann mehrere Male hintereinander mit seinem Schwert angreifen und sich mit dem Schild verteidigen. Die besondere Fähigkeit ist der Kampfschrei, womit er seine Gegner für kurze Zeit verwirren und aufhalten kann.
	\subsection{Kämpfer}
	Seine einzige Waffe ist seine Hände und somit für den Nahkampf besonders gut, da er auch dein einen oder anderen K.O.- Schlag verpassen kann.
	\chapter{Ziel}
	Um den an den Preis zu kommen, braucht man eine Schlüssel. Dieser ist in drei  geteilt. Jedes Team und der Game Master besitzen einen Teil davon. Um an die anderen Stücke zu gelangen, müssen die Teams sich gegenseitig angreifen. Wird das eine Team besiegt, bekommt man das andere Teil. Die besiegten Spielfiguren können jedoch wiederbelebt werden und weiterhin für die Schlüsselteile kämpfen. Hat man den kompletten Schlüssel, gelangt man an den Schatz und gewinnt das Spiel.
\end{document}