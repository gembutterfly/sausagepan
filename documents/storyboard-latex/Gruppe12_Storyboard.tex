\documentclass[10pt,a4paper,notitlepage]{report}
\usepackage[utf8]{inputenc}
\usepackage{amsmath}
\usepackage{amsfonts}
\usepackage{amssymb}
\usepackage{graphicx}
\usepackage{xcolor}
\usepackage{geometry}
\geometry{a4paper, top=15mm, left=25mm, right=25mm, bottom=25mm, headsep=10mm, footskip=10mm}
\pagestyle{empty} %keine Kopf-/Fußzeile
\author{Sausage Pan}
\begin{document}
	%Farbdefinierung
	\definecolor{orange}{HTML}{F67800}
	\definecolor{hellorange}{HTML}{FFAD41}
	\definecolor{schwarz}{rgb}{0,0,0}
	%Stildefinitionen!!Wichtig!!
	\newcommand{\Eins}[1]{\color{orange}\textbf{{\Large#1}}} %Überschrift 1. Ordnung
	\newcommand{\Zwei}[1]{\color{orange}\textbf{{\large#1}}} %Überschrift 2. Ordnung
	\newcommand{\Drei}[1]{\color{orange}{\normalsize#1}} %Überschrift 3. Ordnung
	\newcommand{\Text}{\color{schwarz}} %normaler Fließtext
	\newcommand{\Fusszeile}
	{\textit{{\footnotesize Eckert, Georg - Roscher, Philipp - Krien, Alexandra - Sinakow, Sergej - Blasberg, Bettina - Groß, Stephanie Sara}}} %Fußzeile immer am Ende der Seite einfügen!
	%Randstreifen
	\marginpar{\vspace{3.0mm} \color{orange}\rule{0.8mm}{53.3mm} \\[3mm] \color{hellorange}\rule{0.8mm}{170mm}}
	%Header-Bild
	\begin{center}
		\includegraphics[width=160mm]{header2}
	\end{center}
	%Eigentlicher Inhalt :D
	\color{orange} \textit{Gruppenname:} \color{schwarz}sausage pan \color{orange} \textit{Gruppenleiter:} \color{schwarz}Georg Eckert   \color{orange} \textit{Protokollant:} \color{schwarz}Philipp Roscher\\
	\\
	\\
	\Eins{Storyboard}\\
	\\
	\Zwei{Titel:} \Text{Prisma}\\
	\\
	\Zwei{Thema:} \Text{Farbenlehre}\\
	\\
	\Zwei{Zielgruppe:} \Text{Schüler der fünften Klasse}\\
	\\
	\Zwei{1. Grundlegender Aufbau des Spiels}\\
	\\
	\Drei{1.1 Setting}\\
	\Text
		Das Spiel bewegt sich in einer fiktiven Welt. Der Spieler übernimmt dabei die Rolle eines namenlosen Helden, 
	welcher weder hinsichtlich des Alters, Geschlechts etc. spezifiziert wird.\\\\
	Ausgangssituation für das Spiel ist, dass in Folge eines Missgeschickes der Regenbogen in Fragmente zerspringt.
	Als Resultat wird die Welt in ein einheitliches Grau getaucht.\\\\
	Aufgabe ist es nun, diese Fragmente wieder einzusammeln um letztendlich den Regenbogen neu zusammensetzen zu können. 
	Diese Fragmente finden sich in den einzelnen Level wieder, die als Jump'n'Run angelegt werden.\\\\
	Der Spieler soll dabei ein grundlegendes Verständnis für die Farbenlehre erhalten. Jedes Level fokussiert sich dabei auf ein anderes Teilthema.\\\\
	Als unterstützende Hilfe steht dem Spieler stets eine Art Mentor zur Verfügung, welcher um Hilfe gebeten werden kann. 
	So wollen wir möglicherweise aufkommenden Frust eindämpfen.\\\\
	Gerne würden wir einige Zwischenlevel einbauen, die andere Spielmechaniken nutzen. 
	Denkbar wären hier kleinere Puzzle oder Quiz, um das Wissen zu festigen. \\\\
	Dies haben wir bisher allerdings als optionalen Inhalt deklariert und wollen uns zunächst um die Hauptlevel kümmern. 
	Je nach verbleibender Zeit werden danach die Zwischenlevel konzipiert.
	\\
	\\
	\clearpage\
	\\
	\Drei{1.2. Spielmechaniken}\\
	\\
	\Text
		Das Spiel wird mit 9 Grundmechaniken auskommen. Die ersten zwei sind genretypisch laufen und springen. 
	Dazu kommt das Verschieben von Gegenständen. Die Farblehre wird durch die Prinzipien der additiven und subtraktiven Farbmischung, 
	sowie der Brechung von Licht vermittelt.\\
	\\
	Das erfordert folgende Spielmechanik:\\
	\begin{itemize}
	\item Einfärben des zu Beginn schwarzen Charakters bei Kontakt mit einer Lichtfarbe in dieselbe\\
	\item additive Farbmischung bei aufeinanderfolgendem Kontakt mit verschiedenen Lichtfarben
	\item Einfärben des Charakters bei Kontakt mit farblichen Flüssigkeiten in dieselbe
	\item subtraktive Farbmischung bei Kontakt mit verschiedenen Flüssigkeiten
	\item Entfernung von Farbpigmenten bei Kontakt mit farbloser Flüssigkeiten
	\item Aufspaltung von Lichtstrahlen in ihr Farbspektrum durch Prismen
	\item Besiegen von Gegnern durch Sprünge bei vorherigem Einfärben mit der Komplementärfarbe
	\item Verdeckung durch Gegenstände, welche vor Lichtstrahlen geschoben werden
	\end{itemize}\
	\\
	\Drei{1.3. Steuerung}\\
	\\
	\Text
		Gespielt wird mit Maus und Tastatur.\\
	Innerhalb der Level kann sich der Spieler über die Pfeiltasten bewegen.
	\clearpage\
	\\
	\Zwei{2. Ablauf}\
	\\
	\\
	\Text
	\underline{Animation 1 - Introsequenz}\
	\\
	\\
		Grundlegende Story:\\
	\begin{flushright}
	\textit{Eine kleine Motte träumt davon einst so schön wie die strahlend bunten Schmetterlinge zu sein. Gedankenversunken bemerkt sie 
	die Ähnlichkeit zwischen den Schmetterlingen und den bunten Farben des Regenbogens. Überzeugt dass dieser auch ihr prächtige Flügel verleihen könnte, 
	fliegt sie zu ihm hinauf. Durch ein Versehen jedoch zerfällt der Regenbogen in Einzelteile und stürzt zu Boden. \\
	Die Welt wird in ein tristes Grau getaucht. Bewusstlos stürzt die Motte gemeinsam mit einem Regenbogenfragment vor die Füße unseres Protagonisten.}
	\end{flushright}\
	\\
	\underline{Level 1}\\\\\
	\textbf{Thema:}\
	Subtraktive Farbmischung
	\\\\
	\textbf{Aufgabe:}\
	missing
	\\\\
	\underline{Animation 2}\
	\\
	\begin{flushright}
	\textit{Die Motte schließt sich dem Protagonisten an und weist ihn zum nächsten Teil, da sie selbst ihr Missgeschick nicht beseitigen kann.\\
	Einführung der Motte als Symbol am unteren Bildschirmrand. Über dieses können von nun an kurze Hilfestellungen abgerufen werden.\\
	Hauptbildschirm als Navigationsfläche. Hier können nun das nächste Level angewählt, eventuelle Achievements angesehen und einige 
	Informationen abgerufen werden.}
	\end{flushright}\
	\\\\
	\underline{Level 2}\\\\\
	\textbf{Thema:}\
	Additive Farbmischung
	\\\\
	\textbf{Aufgabe:}\
	Protagonist befindet sich wieder in einem optisch ähnlichen Jump n Run Level.\\
	Verschiedene Lichtkegel die in Grundfarben leuchten, befinden sich in Sichtweite, ebenso ein Tor in einer Mischfarbe.
	 Der Spieler muss nun die Figur so durch das Level lotsen, dass sich die Figur in der gleichen Farbe wie das Tor einfärbt. 
	Dazu muss er die Figur durch die unterschiedlichen Lichtkegel führen, so dass die Mischfarbe der betretenen Lichtfarben die Gesuchte ergibt.\\
	\\\\
	\underline{Level 3}\\\\\
	\textbf{Thema:}\
	Absorption
	\\\\
	\textbf{Aufgabe:}\
	missing
	\\\\
	\underline{Level 4}\\\\\
	\textbf{Thema:}\
	Komplementärfarben
	\\\\
	\textbf{Aufgabe:}\
	missing
	\\\\
	\underline{Level 5}\\\\\
	\textbf{Thema:}\
	Warme/Kalte Farben
	\\\\
	\textbf{Aufgabe:}\
	missing
	\\\\
	\underline{Level 6}\\\\\
	\textbf{Thema:}\
	Lichtbrechung
	\\\\
	\textbf{Aufgabe:}\
	missing
	\\\\
	\underline{Level 7}\\\\\
	\textbf{Thema:}\
	Helligkeit
	\\\\
	\textbf{Aufgabe:}\
	missing
	\\
	\clearpage\
	\\
	\Zwei{3. Optionale Inhalte}\
	\\\\
	\Text
		Da wir noch nicht wirklich abschätzen können, wie viel Zeit unser Projekt tatsächlich in Anspruch nehmen wird, 
	beschränken wir uns bisher nur auf die wirklich nötigen Spielfunktionen. \\
	Gerne würden wir diese später aber um einige Features erweitern.
	\\\\
	\Drei{3.1. Minigames}\
	\\\\
	\Text
		Um das Erlernte zu festigen, noch einmal anzuwenden oder vielleicht auch erst verständlich zu machen, wäre es denkbar einige Minigames einzubauen.\\ 
	Bei diesen soll es sich um von der Motte gestellte Aufgaben handeln, die mit fortschreitendem Handlungsbogen freigeschaltet werden 
	und thematisch gleich mit den Level sind.\\
	Im Gegensatz zu den Jump n Run Level soll hier weniger Intuition gefragt sein. Der Spieler hat die Möglichkeit sich Themen von der Motte 
	erklären zu lassen und kann dieses Wissen schließlich auch in den regulären Level anwenden.\\
	Denkbare Spielprinzipien wären Quizfragen, kleinere Rätsel, Zuordnungen per Drag n Drop etc.\\
	Für das Lösen aller Minigames wird schließlich das letzte Level freigegeben.\\
	\\
	\Drei{3.2. Achievements}\
	\\\\
	\Text
		Um einen Anreiz für das erfolgreiche Beenden des Spiels zu schaffen könnte man Achievements anlegen.\\
	Diese können dann teilweise durch den regulären Spielverlauf, aber auch durch besondere Aktionen freigeschaltet werden.\\
	\\
	\Drei{3.3. Eastereggs}
	\\\\
	\Text
		Um auch selbst ein bisschen Spaß an dem Spiel zu haben, wäre es toll einige Spielereien einzubauen sofern uns die Zeit dafür bleibt.\\
	\\
	\Fusszeile
\end{document}