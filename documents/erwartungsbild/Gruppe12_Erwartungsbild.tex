\documentclass[10pt,a4paper,notitlepage]{report}
\usepackage[utf8]{inputenc}
\usepackage{amsmath}
\usepackage{amsfonts}
\usepackage{amssymb}
\usepackage{graphicx}
\usepackage{xcolor}
\usepackage{geometry}
\usepackage{picinpar}
\geometry{a4paper, top=15mm, left=25mm, right=25mm, bottom=25mm, headsep=10mm, footskip=10mm}
%\pagestyle{empty} %keine Kopf-/Fußzeile
\author{Sausage Pan}
\begin{document}
	%Farbdefinierung
	\definecolor{orange}{HTML}{F67800}
	\definecolor{hellorange}{HTML}{FFAD41}
	\definecolor{schwarz}{rgb}{0,0,0}
	%Stildefinitionen!!Wichtig!!
	\newcommand{\Eins}[1]{\color{orange}\textbf{{\Large#1}}} %Überschrift 1. Ordnung
	\newcommand{\Zwei}[1]{\color{orange}\textbf{{\large#1}}} %Überschrift 2. Ordnung
	\newcommand{\Drei}[1]{\color{orange}{\normalsize#1}} %Überschrift 3. Ordnung
	\newcommand{\Text}{\color{schwarz}} %normaler Fließtext
	\newcommand{\Fusszeile}
	{\textit{{\footnotesize Eckert, Georg - Roscher, Philipp - Krien, Alexandra - Sinakow, Sergej - Blasberg, Bettina - Groß, Stephanie Sara}}} %Fußzeile immer am Ende der Seite einfügen!
	%Randstreifen
	\marginpar{\vspace{3.0mm} \color{orange}\rule{0.8mm}{53.3mm} \\[3mm] \color{hellorange}\rule{0.8mm}{170mm}}
	%Header-Bild
	\begin{center}
		\includegraphics[width=160mm]{header2}
	\end{center}
	%Eigentlicher Inhalt :D
	\Eins{Erwartungsbild}\\\
\\
	\Text
	Unser Prototyp beschränkt sich auf das Startmenü, die Introsequenz und das erste Level. Die Levelübersicht wurde in Mangel der hinführenden Zwischensequenz 
	zunächst rausgelassen. Sie ist jedoch zumindest bildlich in der Introsequenz zu sehen, als sie sich entfärbt.\\
	\\
	Die Introsequenz soll den Spieler an die Thematik ranführen. Die genaue Geschichte dahinter ist dem Storyboard zu entnehmen.\\
	Es ist dabei Absicht, dass der Spieler noch nicht weiß, welche Rolle er dabei in diesem Szenario einnimmt.
	Daher ist auch das erste Level sehr einfach gehalten. Es dient ins Besondere dazu, den Spieler an die allgemeine Steuerung heranzuführen und das erste
	Thema 'Pigmentfarbmischung' zu vermitteln.\\
	Es wird dabei eine gewisse Neugier vorausgesetzt, die den Spieler dazu antreibt das Szenario zu erkunden.
	Schnell sollte er dabei erkennen, dass er seinen Charakter einfärben kann. Sollten die nötigen Kenntnisse zur Farbmischung fehlen,
	kann weiterhin ein Hilfefenster mit einigen Hinweisen geöffnet werden. Eine geeignete Heranführung an dieses Hilfemenü
	ist im Prototyp noch nicht enthalten.\\
	Ziel des Levels ist den Charakter in der gleichen Farbe wie das Tor einzufärben und damit das Level zu beenden.
	Mit Durchschreiten des Tores erhält der Spieler das erste Regenbogenfragment, an welchem sich auch die bewusstlose Motte befindet.
	In einer anschließenden (noch fehlenden)  Animationssequenz erklärt die Motte das Ziel des Spiels: alle Regenbogenfragemente zu finden und bittet ihn damit um
 	seine Hilfe.\\
	Anschließend wird auf die Levelübersicht weitergeleitet.\\
	\\
	Ein weiterer Aufgabentyp ist ebenfalls noch nicht integriert. Jedoch haben wir bereits begonnen ein kleines Minispiel zu entwickelen, in dem der Spieler 
	über Lampen weiter mit der additiven Farbmischung vertraut gemacht wird. Im ähnlichen Stil würden wir gern einige Quiz und Drag'n'Drop Minispiele
	einbauen.\\
	Diese sollen dazu dienen das Erlernte weiter zu festigen.\\
	\\
	Fehlerkontrolle im klassischen Sinne findet nicht statt. Gern würden wir den Spieler beim erstmaligen Versuch das Tor zu durchschreiten darauf hinweisen,
	dass er sich in seiner Farbe an das Tor anpassen muss. Färbt sich der Spieler allerdings im Spielverlauf zunächst falsch, wird darauf nicht hingewiesen.
	Es soll spielerisch erlernt werden welche Farbkombination zum Ziel führt.\\
	\\\\
	\Fusszeile
\end{document}